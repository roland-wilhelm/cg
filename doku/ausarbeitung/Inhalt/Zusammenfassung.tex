\section{Fazit}
\label{sec:Fazit}

\subsection{Zusammenfassung}
Die Aufgaben konnten alle gelöst werden, allerdings erforderten gerade die ersten drei Aufgaben sehr viel Zeit. Insgesamt musste man teilweise sehr viel Zeit für Dinge aufwenden, die nicht direkt mit der eigentlichen Aufgabe zu tun haben. Beispielsweise war das parsen des Input Files in der zweiten Aufgabe extrem aufwendig, wenn man nicht Java verwendet. In Aufgabe 3 war die suche nach Funktionen der STL, die man verwenden kann und die auch funktionieren wie erwartet sehr aufwendig. Die eigentliche Aufgabe war dagegen weniger komplex.

Vermutlich hätte man sich viel Arbeit sparen können, wenn die Zusammenarbeit zwischen den Teams besser gewesen wäre. Man hätte dann gemeinsame Parser oder gleiche STL-Funktionen verwenden können. Tatsächlich wurden allerdings nur die Ergebnisse verglichen und teilweise die Algorithmen, allerdings erst nachdem man selbst eine Implementierung konstruiert hat.

Insgesamt hat das Praktikum sehr zum Verständnis und der Vertiefung des Vorlesungsstoff beigetragen. Durch die Aufgaben musste man sich mit den Problemen weiter auseinandersetzten und sich funtionnierende Konzepte für die Realisierung überlegen. Selbst wenn die meisten Probleme weniger mit dem Stoff, als mit der Umsetzung in Programmcode zu tun hatten, ist der Rückblick trotzdem positiv.

\subsection{Lessons Learned}
Wir haben während des Praktikums gelernt, dass man durch Kooperation zu den besten Ergebnissen kommt. Durch die zusätzliche Kreativität und Inspiration, die man durch Diskussionen mit anderen Teams bekommt, gewinnt auch die Eigene Arbeit an Qualität.

Aufgabe 1 hat verdeutlicht, dass es nicht einen richtigen Algorithmus gibt. Wir haben es geschafft in einem Team 2 verschiedene Realisierungen zu Implementieren. Beide Ideen haben ihre Vor- und Nachteile und da wir uns nicht einigen konnten, welcher nun besser ist haben wir uns entschieden beide unabhängig zu verwenden.

Sowohl die Funktion des Line Sweep, als auch die STL-Funktionen zu Listen und Vectoren wurden in Aufgabe 3 verwendet und dadruch vertieft. In dieser Aufgabe konnte man erkennen, dass lineare Laufzeiten schlechter sein können als quadratische, da man die Art der Faktoren beachten muss. Bei Outputsensiven Algorithmen muss beachtet werden ob der Input einen sinnvollen Einsatz zulässt.