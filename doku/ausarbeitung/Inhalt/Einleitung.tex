\section{Einleitung}
\label{sec:Einleitung}

Diese Studienarbeit beschreibt das Praktikum zur Vorlesung Embedded- und Echtzeitbetriebssysteme. Die Studenten sollen darin
den Umgang mit Betriebssystemen von Embedded Systemen kennenlernen. Die grundlegenden Werkzeuge im Praktikum sind ein BeagleBoard, 
das Echtzeitbetriebssystem QNX und die dazugeh�rige Entwicklungsumgebung QNX Momentics.
Auf dieser Basis werden im Verlauf mehrere Aufgaben erarbeitet. Diese vertiefen die Themen der Vorlesung und gehen auf spezielle 
Sachverhalte intensiver ein. Ziel ist es zyklisch Threads zu starten und mit Hilfe einer selbstentwickelten zeitverschwende Funktion 
das Embedded System auszulasten. Diese Auslastung wird mit Hilfe von Momentics dargestellt und es kann analysiert werden, ob die 
Threads ihre Echtzeitbedingungen einhalten k�nnen. Abschlie�end wird der QNX Kernel noch optimiert, sodass er lediglich die unbedingt
n�tigen Module enth�lt.
