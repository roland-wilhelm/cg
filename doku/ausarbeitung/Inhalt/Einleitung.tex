\section{Einleitung}
\label{sec:Einleitung}

Diese Studienarbeit beschreibt das Praktikum zur Vorlesung Computational Geometry. Die Studenten sollen darin den Umgang mit Mathematischen Problemen in Zusammenhang mit Computern oder Robotern lernen. Daf�r sollen relativ einfache Probleme mit einer beliebigen Programmiersprache und selbst erstellten Algorithmen gel�st werden. Im Praktikum werden nur die Aufgaben und Testdaten zur Verf�gung gestellt, weitere Werkzeuge gibt es nicht. 

Auf dieser Basis werden im Verlauf mehrere Aufgaben erarbeitet. Diese vertiefen die Themen der Vorlesung und gehen auf spezielle Sachverhalte intensiver ein. Ziel ist es meistens Schnittpunkte, ihre Anzahl und Fl�chen zu bestimmen. Das Praktikum zeigt, dass es beliebig komplex werden kann triviale Mathematische Probleme in performante Algorithmen zu konvertieren.
